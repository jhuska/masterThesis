% fithesis2 with modifications used, please use local fithesis.cls file, not system-wide installed.
\documentclass[11pt,oneside,final]{fithesis2}
% \documentclass[oneside,final]{fithesis2}
% \usepackage[resetfonts]{cmap}
\usepackage{lmodern}

\usepackage[english]{babel}
\usepackage[utf8]{inputenc}
\usepackage[T1]{fontenc}

\usepackage{hyperref}
\usepackage{graphicx}
\usepackage{color}
\usepackage{afterpage}
\usepackage{calc}
\usepackage{subfig}
\usepackage{amssymb}
\usepackage{amsthm}
\usepackage{amsmath}
\usepackage{float}
\restylefloat{figure}

\usepackage{fixltx2e}

\def\R{\mbox{\sffamily\bfseries R}}

\DeclareGraphicsExtensions{.pdf,.png,.jpg,.gif}

\thesislang{en}
\thesistitle{Visual testing something catchy}
\thesissubtitle{Diploma thesis}
\thesisstudent{Juraj Húska}
\thesiswoman{false}
\thesisfaculty{fi}
\thesisyear{2015}
\thesisadvisor{Mgr.\,Marek Grác,\,Ph.D.}

\newcommand{\reci}[1]{\frac{1}{#1}}
\newcommand{\hypot}[2]{\sqrt{#1^2+#2^2}}
\newcommand{\cbrt}[1]{\sqrt[3]{#1}}

% protocols & commands
\newcommand{\comproto}[1]{\emph{#1}}
\newcommand{\protocommand}[1]{\emph{\uppercase{#1}}}
\newcommand{\protoparam}[1]{\emph{#1}}

% some math & modulo
\newtheorem{mydef}{Definition}
\newtheorem{myprop}{Proposition}
\newtheorem{mytheorem}{Theorem}
\makeatletter
\def\imod#1{\allowbreak\mkern10mu({\operator@font mod}\,\,#1)}
\makeatother

\newcommand{\gfe}{\ensuremath{\text{GF}\left(2^8\right)}}
\newcommand{\gf}{\ensuremath{\text{GF}\left(2\right)}}

% bibtex
% Czech bibtex citation norms
% http://www.abclinuxu.cz/blog/Drobnosti/2007/3/csplainnat.bst-nbsp-cesky-styl-pro-bibtex-dle-iso-nbsp-690
% svn://kraken.pedf.cuni.cz/csplainnat/
% http://www.fit.vutbr.cz/~martinek/latex/czechiso.html.cs.iso-8859-2
% http://repo.or.cz/w/csplainnat.git
% http://www.root.cz/clanky/odborny-text-v-lyx-matematika-a-bibliografie/nazory/418552/
\usepackage{url}
\usepackage[numbers]{natbib}
\bibliographystyle{unsrtnat}
%\bibliographystyle{plain}

\usepackage{fancyhdr}
\pagestyle{plain}

% multi-row
%\usepackage{multirow}
\usepackage{color, colortbl}

\definecolor{Gray}{gray}{0.85}
\newcommand{\clg}{\cellcolor{Gray}}
\newcommand{\eal}{\emph{et~al.}}

% \fancyhead[LE,RO]{\slshape \rightmark}
% \fancyhead[LO,RE]{\slshape \leftmark}
% \fancyfoot[C]{\thepage}


\hyphenation{how-to}

\begin{document}


\newenvironment{atribut_description}
{\begin{description}
  \renewcommand{\makelabel}[1]{\texttt{\hspace{6pt}##1 $-$}}%
  \setlength{\itemsep}{1pt}
  \setlength{\parskip}{0pt}
  \setlength{\parsep}{0pt}}
{\end{description}}
\renewcommand{\tiny}{\fontsize{7.7}{9.7}\selectfont}

\FrontMatter
\ThesisTitlePage

\begin{ThesisDeclaration}
\DeclarationText
\AdvisorName
\end{ThesisDeclaration}

\begin{ThesisThanks}
Some people helped me a lot and some not at all. Nevertheless, I would like to thank all.
\end{ThesisThanks}

\begin{ThesisAbstract}
This thesis is very important!
\end{ThesisAbstract}
 
\begin{ThesisKeyWords}
key word1, and so on
\end{ThesisKeyWords}
\MainMatter
\renewcommand{\contentsname}{Table of contents}

\tableofcontents

\chapter{Introduction}    
There is a big demand for this thesis.
Need and cost of manual testing, space for improvement.
    
\chapter{Visual testing of software}    
    Testing of software in general is any activity aimed at evaluating an attribute or capability of a program and determining that it meets its required results [1]. 
    It can be done either manually by actual using of an application or automatically by executing testing scripts.
    
    Visual testing of an application is an effort to find out its non-functional errors, which expose themselves by changing a graphical state of an application under test.

  \section{Visual testing in release testing process}
  Nowadays software is often released for general availability in repetitive cycles, which are defined according to a particular software development process such as Waterfall [2], or Scrum [3].
      
  \section{Need for automation}
  comparison of hiring people to do manual testing vs. automated testing cost
    
\chapter{Analysis of existing solutions}
How the process of testing with these tools looks like, its advantages and disadvantages.
  
  \section{Mogo}
  
  \section{BBC Wraith}
  
  \section{PhantomCSS}
  
  \section{Facebook Huxley}
  
  \section{Rusheye}
  
  \section{Drawbacks}
  Conclusion of drawbacks, and why we try to propose another approach
  
\chapter{New approach}
  
  \section{Hypothesis}
  Simply: reuse of functional tests of the application for visual testing
  
  \section{Process}
  How one would use my tool and where in testing stack such visual testing has its place, written in business process notation
  
  \section{Analysis of useful tool output}
  Requirements for useful output of such a tool based on questionnaire for RichFaces team, or maybe I will ask all JBoss employees
  
\chapter{Implemented tool}
An answer to the new process, requirements: CI viable, reusing what can be reused, extensible, cloud ready, multiple users

  \section{Client part}
  
    \subsection{Arquillian}
    Integration testing, starting containers, event based machine
  
    \subsection{Arquillian Graphene}
    Functional testing of Web UI, screenshooter
  
    \subsection{Rusheye}
    Screenshots comparison, rewritten to Arquillian core
  
    \subsection{Graphene visual testing}
    An adaptor between Rusheye and Arquillian Graphene
  
  \section{Server part}
  
    \subsection{Web application to view results}
    Its architecture, reasoning for chosen solutions, screenshots of app, key functionality
    
    \subsection{Storage of patterns}
    Description of solution, reasoning
    
\chapter{Deployment of tool and process}

  \section{Deployment on production application}
  Deployment on stable app
  
  \section{Deployment on development application}
  Deployment sooner on application which is in Alpha phase, my hypothesis is that it will not be worth to deploy it on such a app, due to too many changes
  
  \section{Usage with CI}
  Jenkins job and its cooperation with the tool, more particullary tool ability to handle multiple jobs, apps, versions, etc.
  
  \section{Cloud ready}
  The app can be easily deployed on Openshift
  
  \section{Results}
  The percentage of improvement of QA effectiveness
  
\chapter{Conclusion}
What I developed, What I improved, What can be better, Possible ways of extensions: Openshift cartridge
    
    % bibtex here
    \addcontentsline{toc}{chapter}{Bibliography}
    \pagestyle{plain}
    \bibliography{thesis}
\end{document}